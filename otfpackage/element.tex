\documentclass[uplatex,tate,book,paper=b6]{jlreq}
\usepackage[noreplace,scale=1]{otf}
\usepackage{bxpapersize}
\newcommand{\?}[1]{#1\hspace{1zw}\ignorespaces}% 文末区切り約物
\renewcommand{\theenumi}{\ajLabel\ajMaruKansuji{enumi}}
\ajCIDVarList{角=13682 浜=8531}
\begin{document}
ユークリッドの『原論』における五つの「公準」とは
要請の集まりのことである。
以下のような「公準」が述べられている。
\begin{enumerate}
\item 任意の☃から他の☃へ\ajHotSpring を引くこと。
\item 有限の\ajHotSpring を連続的に真っ直ぐに伸ばすこと。
\item 任意の\CID{7652}飾区と\CID{1481}城市で\CID{20957}を描くこと。
\item 全ての\ajLig{オングストローム*}は等しいこと。
\item \ajHotSpring が二\ajHotSpring と交わり、
  同じ側の内\ajVar{角}の和が
  二\ajLig{オングストローム*}未満の場合、
  その二\ajHotSpring が限りなく伸ばされると、
  内\ajVar{角}の和が二\ajLig{オングストローム*}より
  小さい側で\□ア\□レ。
\end{enumerate}
アレッ\?{\ajLig{!?}}
もしかしたらチョット違ってるかもしれない。
\end{document}

